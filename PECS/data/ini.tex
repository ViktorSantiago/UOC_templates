\cabeceraNum{1}
	\enu{1}{El primer tipo de caja contiene enunciados que vienen numerados; pero sin indicar el apartado.}

	\begin{itemize}
		\item[A)]{En realidad no importa mucho}
		\item[B)]{porque $\LaTeX$ permite}
		\item[C)]{crear listas numeradas por letras (o números, o números romanos)}
		\item[D)]{y con ella crear los sub-apartados}

	\end{itemize}

	Así que se puede usar éste tipo de enunciado para preguntas en las que los apartados no sean demasiado independientes entre sí.

	\enunciado{2}{A}{El segundo tipo de caja contiene enunciados numerados y con la indicación del número de apartado.}

	Si los apartados de la pregunta son ejercicios en sí mismos, podemos usar éste tipo de caja que incluye la indicación del apartado dentro de la pregunta. 

	Las \texttt{PEC} suelen estar numeradas, así que la cabecera del documento nos servirá. Pero en algunas asignaturas sólo hay una {\bf práctica}. No tiene mucho sentido numerarla, por lo que podemos usar la {\it cabecera no numerada} en su lugar: 

	\cabecera

	Pero nunca perderemos el poder de $\LaTeX$ :

	\begin{equation*}
  	x = a_0 + \cfrac{1}{a_1
    	      + \cfrac{1}{a_2
      	    + \cfrac{1}{a_3 + \cfrac{1}{a_4} } } }
	\end{equation*}

	¡Y hay más!:

\begin{minted}[linenos]{c++}
  #include <iostream>
  using namespace std;

  int main() {
    cout << "Hola Mundo" << endl;
    return 0;
  }
	\end{minted}